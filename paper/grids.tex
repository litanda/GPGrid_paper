\section{Theoretical  model grid}\label{sec:grid}

We built up a model grid as the base data of GP model. In this work, we aimed cover stars with approximate solar mass on main-sequence and subgiant phases. Thus, the mass range is set up as 0.8 -- 1.2$M_{\odot}$, and we computed each stellar evolutionary track from the Hayashi line and to the base of red-giant branch where $/log g$ = 3.5 dex. The gird includes four independent model inputs: stellar mass (M), initial helium fraction ($Y_{\rm init}$), initial metallicity ([Fe/H]), and the mixing-length parameter ($\alpha_{\rm MLT}$).  Ranges and grid steps of the four model inputs are summarized in Table \ref{tab:grid}.

\begin{table*}
	\centering
	\caption{Stellar model computations for training and validating sets.}
	\label{tab:grid}
	\begin{tabular}{llll} % four columns, alignment for each
		\hline
		\multicolumn{4}{c}{Training model set (Grid-based)}\\
		\hline
		Input Parameter & Range & Increment & $N_{\rm track}$\\
        \hline
	$M$ [$M_{\odot}$]  & 0.80 -- 1.20 &  0.01&\\
        $\rm{[Fe/H]}$ [dex] & -0.5 -- 0.2/0.2 -- 0.5 & 0.1/0.05 & 15,375\\
        	$Y_{\rm init}$ & 0.24 -- 0.32 & 0.02&\\
        $\alpha_{\rm{MLT}}$  & 1.7 -- 2.5&  0.2&\\
        \hline
        \multicolumn{3}{c}{Validating model set (Randomly computed in above parameter ranges)}\\
	\hline
	GPR model & Varying parameters & Fixed parameters& $N_{\rm track}$\\
	\hline
	2D& $M$, $t_{\rm frac}$ & $\rm{[Fe/H]}$ = 0.0,  $Y_{\rm init}$ = 0.28, $\alpha_{\rm{MLT}}$  = 2.1&15\\
	3D& $M$, $t_{\rm frac}$, $\rm{[Fe/H]}$ &$Y_{\rm init}$ = 0.28, $\alpha_{\rm{MLT}}$  = 2.1&200\\
	4D& $M$, $t_{\rm frac}$, $\rm{[Fe/H]}$, $Y_{\rm init}$ & $\alpha_{\rm{MLT}}$  = 2.1&1,000\\
	5D& $M$, $t_{\rm frac}$, $\rm{[Fe/H]}$, $Y_{\rm init}$, $\alpha_{\rm{MLT}}$ & - &4,000\\
        \hline

        %Other physics & \multicolumn{2}{Scheme}\\
        %\hline
        %Diffusion & \multicolumn{2}{Yes}\\
        %Overshooting & \multicolumn{2}{N/A}\\
        %\hline
%     \multicolumn{3}{p{.4\textwidth}}{$^{\rm a}Y_0$ = 0.249 and $\frac{\Delta Y}{\Delta Z}$ = 1.33 was adopted.}  
	\end{tabular}
\end{table*}

\subsection{Stellar models and input physics}\label{subsec:stellar_model}

We used Modules for Experiments in Stellar Astrophysics
(\textsc{MESA}, version 12115) to establish a grid of stellar models. 
\textsc{MESA} is an open-source stellar evolution package which is undergoing active development. Descriptions of input physics and numerical methods
can be found in \citet{2011ApJS..192....3P,2013ApJS..208....4P, 2015ApJS..220...15P}.
We adopted the solar chemical mixture [$(Z/X)_{\odot}$ = 0.0181]
provided by \citet{2009ARA&A..47..481A}. 
The initial chemical composition was calculated by: 
\begin{equation}
\log (Z_{\rm{init}}/X_{\rm{init}}) = \log (Z/X)_{\odot} + \rm{[Fe/H]}.  \\
\end{equation}
We used the \textsc{MESA} $\rho-T$ tables based on the 2005
update of OPAL EOS tables \citep{2002ApJ...576.1064R} and OPAL opacity
supplemented by low-temperature opacity \citep{2005ApJ...623..585F}. 
The MESA ‘simple’ photosphere were used as the set of boundary conditions for modelling the atmosphere.
The mixing-length theory of convection was implemented, where 
$\alpha_{\rm MLT} = \ell_{\rm MLT}/H_p$ is the mixing-length parameter.
We also applied the \textsc{MESA} predictive mixing scheme \citep{2018ApJS..234...34P,2019ApJS..243...10P}  in the model computation.  The \textsc{MESA} inlist used for the computation is available on \url{https://github.com/litanda/mesa_inlist}.  


The evolution time step was mainly controlled by the set-up tolerances on changes in surface effective temperature and luminosity.
We saved one structural model at every time step at main sequence and every two steps after central hydrogen exhaustion. For each evolutionary track, we obtained $\sim$100 at the main-sequence stage and $500-700$ at evolved stages.  

\subsection{Oscillation models and seismic $\Delta \nu$}\label{subsec:seismo_model}

Theoretical stellar oscillations were calculated with the \textsc{GYRE} code (version 5.1), which was developed by \citet{2013MNRAS.435.3406T}. And we computed radial modes (for $\ell$ = 0) by solving the adiabatic stellar pulsation equations with the structural models generated by \textsc{MESA}. We computed a seismic large separation($\Delta \nu$) for each model with theoretical radial modes to avoid the systematic offset of the scaling relation. We derived $\Delta \nu$ with the approach given by \citet{2011ApJ...743..161W}, which is a weighted least-squares fit to the radial frequencies as a function of $n$.  




