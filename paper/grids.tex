\section{Theoretical stellar grid}\label{sec:grid}

We compute a stellar model grid as the training dataset. We aim to cover stars with approximate solar mass on the main-sequence and the subgiant phases. The mass range is set up as 0.8 -- 1.2$\rm M_{\odot}$. The computation of evolutionary tracks starts at the Hayashi line and terminates at the base of red-giant branch (RGB) where $\log g$ = 3.6 dex. Note that we only use models after the zero-age-main-sequence (ZAMS). We define ZAMS as the point where core-hydrogen burning contributes over 99.9\% of the total luminosity. 
%
The stellar gird considers four independent fundamental inputs which are stellar mass ($M$), initial helium fraction ($Y_{\rm init}$), initial metallicity ([Fe/H]$_{\rm init}$), and the mixing-length parameter ($\alpha_{\rm MLT}$). 
%
We calculated three model grids. First, a primary grid covers the whole input range. Uniform grid step is applied for $M$, $Y_{\rm init}$, $\alpha_{\rm MLT}$, and we use tow different grid steps for [Fe/H]$_{\rm init}$ below and above 0.2 dex.  Second, an additional grid is computed for $M$ > 1.05$\rm M_{\odot}$. Grid points of this grid are in between of the primary grid to increase the resolution for tracks with the 'hook'. 
Third, we compute off-grid models with random fundamental input values as an independent dataset for validating and testing GP models. 
%4,880 tracks with input parameters that are randomly sampled in the grid ranges for validating GPR models. The evolution time step was mainly controlled by the set-up tolerances on changes in surface effective temperature and luminosity. We also saved structural models for computing theoretical oscillation models.
%
Details of the computation are listed in Table \ref{tab:grid}. 

\begin{table}
	\centering
	\caption{Computation of Stellar model grid.}
	\label{tab:grid}
	\begin{tabular}{llll} % four columns, alignment for each
		\hline
		\multicolumn{3}{c}{Primary Grid}\\
		\hline
		Input Parameter & Range & Increment \\
        \hline
	$M$ ($\rm M_{\odot}$) & 0.80 -- 1.20 &  0.01\\
        $\rm{[Fe/H]}$ (dex) & -0.5 -- 0.2/0.2 -- 0.5 & 0.1/0.05\\
        	$Y_{\rm init}$ & 0.24 -- 0.32 & 0.02\\
        $\alpha_{\rm{MLT}}$  & 1.7 -- 2.5&  0.2\\
        \hline
       \multicolumn{3}{c}{Additional Grid}\\
	\hline
	Input Parameter & Range & Increment \\
        \hline
	$M$ ($\rm M_{\odot}$)  & 1.055 -- 1.195 &  0.01\\
        $\rm{[Fe/H]}$ (dex) & 0.25 -- 0.45 & 0.1\\
        	$Y_{\rm init}$ & 0.25 -- 0.31 & 0.02\\
        $\alpha_{\rm{MLT}}$  & 1.8-- 2.4&  0.2\\
        \hline
        \multicolumn{3}{c}{Off-grid Models}\\
        \hline
        \multicolumn{3}{c}{Input Parameters} &N\\
        \hline
         \multicolumn{3}{l}{Random $M$, [Fe/H]$_{\rm init}$ = 0.0, $Y_{\rm init}$ = 0.28, $\alpha_{\rm MLT}$ = 2.1 } & 44\\
        \multicolumn{3}{l}{Random $M$ and [Fe/H]$_{\rm init}$, $Y_{\rm init}$ = 0.28, $\alpha_{\rm MLT}$ = 2.1 }&174\\
        \multicolumn{3}{l}{Random $M$, [Fe/H]$_{\rm init}$, $Y_{\rm init}$, and $\alpha_{\rm MLT}$}&4880\\
	\hline
	\end{tabular}
\end{table}

%\subsection{Stellar models and input physics}\label{subsec:stellar_model}

We use the stellar code Modules for Experiments in Stellar Astrophysics
(\textsc{MESA}, version 12115) to construct stellar grids. 
\textsc{MESA} is an open-source stellar evolution package which is undergoing active development. 
Descriptions of input physics and numerical methods
can be found in \citet{2011ApJS..192....3P,2013ApJS..208....4P, 2015ApJS..220...15P}.
We adopted the solar chemical mixture [$(Z/X)_{\odot}$ = 0.0181]
provided by \citet{2009ARA&A..47..481A}. 
The initial helium fraction ($Y_{\rm init}$) and initial metallicity ($\rm{[Fe/H]_{init}}$) are independent inputs. 
The initial chemical composition is calculated with 
\begin{equation}
\log (Z_{\rm{init}}/X_{\rm{init}}) = \log (Z/X)_{\odot} + \rm{[Fe/H]_{init}}.  \\
\end{equation}
We use the \textsc{MESA} $\rho-T$ tables based on the 2005
update of OPAL EOS tables \citep{2002ApJ...576.1064R} and OPAL opacity
supplemented by low-temperature opacity \citep{2005ApJ...623..585F}. 
The grey Eddington $T-\tau$ relation is used to determine boundary conditions for modelling the atmosphere.
The mixing-length theory is implemented and the convection is adjusted by the mixing-length parameter ($\alpha_{\rm MLT}$).
We also apply the \textsc{MESA} predictive mixing scheme \citep{2018ApJS..234...34P,2019ApJS..243...10P}, which improves model structures at the convective boundary.  
Atomic diffusion of helium and heavy elements was also taken into account. MESA calculates particle diffusion and gravitational settling by solving Burger's equations using the method and diffusion coefficients of \citet{Thoul94}. 
%We considered 8 classes of species (H1, He3, He4, C12, N14, O16, Ne20, and Mg24) We considered 8 classes of species (H1, He3, He4, C12, N14, O16, Ne20, and Mg24)
We consider eight elements (${}^1{\rm H}, {}^3{\rm He}, {}^4{\rm He}, {}^{12}{\rm C}, {}^{14}{\rm N}, {}^{16}{\rm O}, {}^{20}{\rm Ne}$, and ${}^{24}{\rm Mg}$)
for diffusion calculations, and have the charge calculated by the MESA ionization module, which estimates the typical ionic charge as a function of $T$, $\rho$, and free electrons per nucleon from \citet{Paquette1986}.
The \textsc{MESA} inlist used for the computation is available on \url{https://github.com/litanda/mesa_inlist}.  

%\subsection{Oscillation models and seismic $\Delta \nu$}\label{subsec:seismo_model}

%Theoretical stellar oscillations were calculated with the \textsc{GYRE} code (version 5.1), which was developed by \citet{2013MNRAS.435.3406T}. And we computed radial modes (for $\ell$ = 0) by solving the adiabatic stellar pulsation equations with the structural models generated by \textsc{MESA}. We computed a seismic large separation($\Delta \nu$) for each model with theoretical radial modes to avoid the systematic offset of the scaling relation. We derived $\Delta \nu$ with the approach given by \citet{2011ApJ...743..161W}, which is a weighted least-squares fit to the radial frequencies as a function of $n$.  


