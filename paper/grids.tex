\section{Theoretical  model grid}\label{sec:grid}

We built up a model grid as the training data for GPR models.  We aimed to cover stars with approximate solar mass on the main-sequence and the subgiant phases. Thus, the mass range is set up as 0.8 -- 1.2$M_{\odot}$, and we computed each stellar evolutionary track from the Hayashi line and terminated at the base of red-giant branch where $\log g$ = 3.6 dex. The gird includes four independent model inputs: stellar mass (M), initial helium fraction ($Y_{\rm init}$), initial metallicity ([Fe/H]), and the mixing-length parameter ($\alpha_{\rm MLT}$). Details of the computations are summarised in Table \ref{tab:grid}. Input parameters of our primary training data is uniformly spaced except the input metallicity, which is double dense above 0.2 $dex$. We also computed an extra grid that fit in between the primary training data for $M > 1.05M_{\odot}$. Because evolutionary tracks in this mass range have the so-called 'hook' at the late main-sequence stage and relatively difficult to model. Lastly, we computed 4,880 tracks with input parameters that are randomly sampled in the grid ranges for validating GPR models. The evolution time step was mainly controlled by the set-up tolerances on changes in surface effective temperature and luminosity. We also saved structural models for computing theoretical oscillation models.

\begin{table*}
	\centering
	\caption{Stellar model computations for training and validating sets.}
	\label{tab:grid}
	\begin{tabular}{llll} % four columns, alignment for each
		\hline
		\multicolumn{4}{c}{Primary Grid}\\
		\hline
		Input Parameter & Range & Increment \\
        \hline
	$M$ [$M_{\odot}$]  & 0.80 -- 1.20 &  0.01&\\
        $\rm{[Fe/H]}$ [dex] & -0.5 -- 0.2/0.2 -- 0.5 & 0.1/0.05\\
        	$Y_{\rm init}$ & 0.24 -- 0.32 & 0.02&\\
        $\alpha_{\rm{MLT}}$  & 1.7 -- 2.5&  0.2&\\
        \hline
       \multicolumn{4}{c}{Extra Grid}\\
	\hline
	Input Parameter & Range & Increment \\
        \hline
	$M$ [$M_{\odot}$]  & 1.055 -- 1.195 &  0.01&\\
        $\rm{[Fe/H]}$ [dex] & 0.25 -- 0.45 & 0.1\\
        	$Y_{\rm init}$ & 0.25 -- 0.31 & 0.02&\\
        $\alpha_{\rm{MLT}}$  & 1.8-- 2.4&  0.2&\\
        %Other physics & \multicolumn{2}{Scheme}\\
        %\hline
        %Diffusion & \multicolumn{2}{Yes}\\
        %Overshooting & \multicolumn{2}{N/A}\\
        \hline
        \multicolumn{4}{c}{Off-grid Models}\\
        \hline
        \multicolumn{4}{c}{4880 tracks with random input parameters }\\
	\hline
	\end{tabular}
\end{table*}

\subsection{Stellar models and input physics}\label{subsec:stellar_model}

We used Modules for Experiments in Stellar Astrophysics
(\textsc{MESA}, version 12115) to establish a grid of stellar models. 
\textsc{MESA} is an open-source stellar evolution package which is undergoing active development. Descriptions of input physics and numerical methods
can be found in \citet{2011ApJS..192....3P,2013ApJS..208....4P, 2015ApJS..220...15P}.
We adopted the solar chemical mixture [$(Z/X)_{\odot}$ = 0.0181]
provided by \citet{2009ARA&A..47..481A}. 
The initial chemical composition was calculated by: 
\begin{equation}
\log (Z_{\rm{init}}/X_{\rm{init}}) = \log (Z/X)_{\odot} + \rm{[Fe/H]}.  \\
\end{equation}
We used the \textsc{MESA} $\rho-T$ tables based on the 2005
update of OPAL EOS tables \citep{2002ApJ...576.1064R} and OPAL opacity
supplemented by low-temperature opacity \citep{2005ApJ...623..585F}. 
The MESA ‘simple’ photosphere were used as the set of boundary conditions for modelling the atmosphere.
The mixing-length theory of convection was implemented, where 
$\alpha_{\rm MLT} = \ell_{\rm MLT}/H_p$ is the mixing-length parameter.
We also applied the \textsc{MESA} predictive mixing scheme \citep{2018ApJS..234...34P,2019ApJS..243...10P}  in the model computation. 
Atomic diffusion of helium and 
heavy elements was also taken into account. MESA calculates particle diffusion and gravitational settling by solving Burger's equations using the method
and diffusion coefficients of \citet{Thoul94}. 
%We considered 8 classes of species (H1, He3, He4, C12, N14, O16, Ne20, and Mg24) We considered 8 classes of species (H1, He3, He4, C12, N14, O16, Ne20, and Mg24)
We considered eight elements (${}^1{\rm H}, {}^3{\rm He}, {}^4{\rm He}, {}^{12}{\rm C}, {}^{14}{\rm N}, {}^{16}{\rm O}, {}^{20}{\rm Ne}$, and ${}^{24}{\rm Mg}$)
for diffusion calculations, and had the charge calculated by the MESA ionization module, which estimates the typical ionic charge as a function of $T$, $\rho$, and free electrons per nucleon from \citet{Paquette1986}.
The \textsc{MESA} inlist used for the computation is available on \url{https://github.com/litanda/mesa_inlist}.  

\subsection{Oscillation models and seismic $\Delta \nu$}\label{subsec:seismo_model}

Theoretical stellar oscillations were calculated with the \textsc{GYRE} code (version 5.1), which was developed by \citet{2013MNRAS.435.3406T}. And we computed radial modes (for $\ell$ = 0) by solving the adiabatic stellar pulsation equations with the structural models generated by \textsc{MESA}. We computed a seismic large separation($\Delta \nu$) for each model with theoretical radial modes to avoid the systematic offset of the scaling relation. We derived $\Delta \nu$ with the approach given by \citet{2011ApJ...743..161W}, which is a weighted least-squares fit to the radial frequencies as a function of $n$.  


