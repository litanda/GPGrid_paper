% mnras_template.tex 
%
% LaTeX template for creating an MNRAS paper
%
% v3.0 released 14 May 2015
% (version numbers match those of mnras.cls)

%%%%%%%%%%%%%%%%%%%%%%%%%%%%%%%%%%%%%%%%%%%%%%%%%%
% Basic setup. Most papers should leave these options alone.
\documentclass[fleqn,usenatbib]{mnras}

% MNRAS is set in Times font. If you don't have this installed (most LaTeX
% installations will be fine) or prefer the old Computer Modern fonts, comment
% out the following line
\usepackage{newtxtext,newtxmath}
% Depending on your LaTeX fonts installation, you might get better results with one of these:
\usepackage{amsmath}
%\usepackage{mathptmx}
%\usepackage{txfonts}
% Use vector fonts, so it zooms properly in on-screen viewing software
% Don't change these lines unless you know what you are doing
\usepackage[T1]{fontenc}

% Allow "Thomas van Noord" and "Simon de Laguarde" and alike to be sorted by "N" and "L" etc. in the bibliography.
% Write the name in the bibliography as "\VAN{Noord}{Van}{van} Noord, Thomas"
\DeclareRobustCommand{\VAN}[3]{#2}
\let\VANthebibliography\thebibliography
\def\thebibliography{\DeclareRobustCommand{\VAN}[3]{##3}\VANthebibliography}


%%%%% AUTHORS - PLACE YOUR OWN PACKAGES HERE %%%%%

% Only include extra packages if you really need them. Common packages are:
\usepackage{graphicx}	% Including figure files
\usepackage{amsmath}	% Advanced maths commands
%\usepackage{amssymb}	% Extra maths symbols

%%%%%%%%%%%%%%%%%%%%%%%%%%%%%%%%%%%%%%%%%%%%%%%%%%

%%%%% AUTHORS - PLACE YOUR OWN COMMANDS HERE %%%%%

% Please keep new commands to a minimum, and use \newcommand not \def to avoid
% overwriting existing commands. Example:
%\newcommand{\pcm}{\,cm$^{-2}$}	% per cm-squared

%%%%%%%%%%%%%%%%%%%%%%%%%%%%%%%%%%%%%%%%%%%%%%%%%%

%%%%%%%%%%%%%%%%%%% TITLE PAGE %%%%%%%%%%%%%%%%%%%

% Title of the paper, and the short title which is used in the headers.
% Keep the title short and informative.
\title[Modelling stars with GP]{Modelling stars with Gaussian Process Regression -- I:  Augmenting Stellar Model Grid}

% The list of authors, and the short list which is used in the headers.
% If you need two or more lines of authors, add an extra line using \newauthor
\author[T. Li et al.]{
Tanda Li,$^{1}$\thanks{E-mail: t.li.2@bham.ac.uk}
Guy R. Davies,$^{1}$\thanks{E-mail: G.R.Davies@bham.ac.uk}
Alex Lyttle,$^{1}$
Lindsey Carboneau,$^{1}$
and A. N. Others$^{1}$
\\
% List of institutions
$^{1}$ School of Physics and Astronomy, University of Birmingham, Birmingham, B15 2TT, United Kingdom\\
}

% These dates will be filled out by the publisher
\date{Accepted XXX. Received YYY; in original form ZZZ}

% Enter the current year, for the copyright statements etc.
\pubyear{2020}

% Don't change these lines
\begin{document}
\label{firstpage}
\pagerange{\pageref{firstpage}--\pageref{lastpage}}
\maketitle

% Abstract of the paper
\begin{abstract}
Grid-based modelling is widely used for estimating stellar parameters. However, stellar model grid is sparse because of the computational cost. This paper demonstrates an application of Gaussian Process (GP) Regression that turns a sparse model grid to a continuous function. We train GP models to map five fundamental inputs (mass, equivalent evolutionary phase, initial metallicity, initial helium fraction, and the mixing-length parameter) to observable outputs (effective temperature, surface gravity, radius, surface metallicity, and stellar age). 
%
We preliminarily test different approaches with a small subset of data and set up the training with the most promising methods. To overcome the limitation of training data size in the GP framework, we section the whole grid and train each section separately. 
%
An off-grid stellar model dataset is then used to test GP predictions. We find no obvious systematic offsets for all five outputs. The median testing error is $\sim$2K for effective temperature, $\sim 0.001$dex for surface gravity, $\sim$ 0.002$\rm R_{\odot}$ for radius, $\sim 0.001$dex for surface metallicity, and $\sim$0.02Gyr for stellar age. However, we find that local systematic uncertainty is not uniform across the parameter space and it mainly varies with mass, equivalent evolutionary phase, and initial metallicity. We hence train another GP model to describe the systematic uncertainties.      
%
We lastly use 100 fake stars to validate the accuracy of GP for modelling stars. GP-determined masses and ages are well consistent with true values within one standard deviation. We also note that GP models give sensible statistical sampling which overcomes the under-sampling issue in the grid-based modelling. 
%
\end{abstract}

% Select between one and six entries from the list of approved keywords.
% Don't make up new ones.
\begin{keywords}
Star: Modelling -- Machine Learning -- keywords
\end{keywords}

%%%%%%%%%%%%%%%%%%%%%%%%%%%%%%%%%%%%%%%%%%%%%%%%%%

%%%%%%%%%%%%%%%%% BODY OF PAPER %%%%%%%%%%%%%%%%%%
%\section{Outline}
This is a temporary section for the outline/to-do list. 
\begin{itemize}
\item[•] Framework of GP: model data augmentation with the established model grid
\item[•] Two strategies: S1 is from observables to fundamentals for modelling stars; S2 is from fundamentals to observables for augmenting model grids. \item[•] Step 1: selection of GP model inputs. Key points include independence of inputs and overlapping of the ranges of base data. 
\item[•] Step 2: selection of GP kernels and validation. 
\item[•] Step 3: uncertainty analysis. Key points include unreliable GP variance and the produce of uncertainty estimating.
\item[•] Step 4: Application of S1: 1) fake stars (MS, turn-off, subgiant) 2) real stars (the Sun, turn-off star, subgiant star) 3) solar calibration
\item[•] Step 5: Application of S2: augmenting the base grid (challenges, limitations due to grid step etc.)
\end{itemize} 


\section{Introduction}

Theoretical stellar model has been developed for decades to simulate star structure and evolution. Star modelling is mostly grid-based \citep[e.g.][]{2016ApJ...823..102C} because computing many stellar models are time-consuming especially when a number of free input parameters are considered. Varying one of these adjusted parameters (mass, metallicity, helium fraction, mixing-length parameter, etc.) adds on an input demission and hence exponentially increases the computational cost. 

A sparse grid is not ideal for the statistics analysis. Classical method like interpolation has been applied to overcome this disadvantage. For instance, \citet{2016ApJS..222....8D} developed a method to transform stellar evolution tracks onto a uniform basis and then interpolate to construct stellar isochrones. More recently, \citet{2019MNRAS.484..771R} uses Bayesian statistics and a Markov Chain Monte Carlo approach to find a representative set of interpolated models from a grid. The interpolation of both works achieve good accuracy for 3-demission girds (inputs are mass, age, and metallicity). However, this approach becomes less reliable for high-demission grid. 
%The algorithm is another {\bf statistically-sound approach to model stars} \citep[e.g. the \textsc{MESA Simplex} module][]{2013ApJS..208....4P}. It offers an automated likelihood minimisation to search for optimal solutions. This method works well for modelling individual stars but is not efficient because the algorithm needs to iteratively compute stellar tracks many time. {\bf Hence, the  algorithm approach is not a good choice for modelling a large sample of stars.} 

Machine learning is being applied to the field of stellar research in many ways to efficiently characterise stars.
\citet{2016MNRAS.461.4206V} applied artificial neural network, which is a series of algorithms that endeavours to recognise underlying relationships in a set of data, to determine the evolutionary parameters of the sun and sun-like stars based on spectroscopic and seismic measurements. Using a similar artificial neural network interference, \citet{2019PASP..131j8001H} developed a method to provide the optimal starting point of model competitions for more detailed forward asteroseismic modelling. Moreover, \citet{2021arXiv210313394M} trained neural networks to predict theoretical pulsation periods of high-order gravity modes, as well as the luminosity, effective temperature, and surface gravity for a given mass, age, overshooting parameter, diffusive envelope mixing, metallicity, and near-core rotation frequency. 
%
Using different machine learning tools, \citet{2016ApJ...830...31B} trained a random forest regressors \citep{ho1995random}, which is an ensemble learning method for regression and operates by constructing a multitude of decision trees, to rapidly estimate fundamental parameters of solar-like stars based on classical and asteroseismic observations. \citet{2018MNRAS.476.3233H} developed a convolutional neural network classifier that analyses visual features in asteroseismic frequency spectra to distinguish between red giant branch stars and helium-core burning stars. \citet{2019MNRAS.484.5315W} determined masses and ages for massive RGB stars from their spectra with a machine-learning method based on kernel principal component analysis, which is a nonlinear form of principal component analysis using integral operator kernel functions and can efficiently compute principal components in high dimensional feature spaces related to input space by some nonlinear map \citep{scholkopf1997kernel}. \citet{2020MNRAS.499.2445H} applied the mixture density network \citep{bishop1994mixture}, which learns a transformation from a set of input variables to a set of output variable, to determine stars' fundamental parameters like mass and age based on observed mode frequencies, spectroscopic, and global seismic parameters.


In above studies, the discriminative machine-learning model is mostly used. The discriminative model treats observables as given facts to directly infer star fundamental parameters. The method is efficient and easy for computation, while the downside is not allowing any priors for star properties like mass.
%
In an opposite direction, the generative machine-learning model uses the star fundamental parameters as given facts to predict observables. This approach offers flexibility to prior fundamental parameters in the sampling. For instance,  \citet{2021MNRAS.tmp.1343L}  determined initial helium fraction and mixing-length parameters for a sample of {\em Kepler} dwarfs and subgiants with an artificial neural network to provide the generative model. This allowed them to prescribe prior distributions over the fundamental stellar parameters and, by extension, over population-level parameters such as a helium enrichment law. Priors encode our current knowledge and assumptions into inference from new data. This is especially important with noisy observations which span a large portion of parameter-space.


Constructing a comprehensive and fine model grid is computationally expensive. In this work, we aim to apply the machine learning tool to transform a sparse model grid onto a continuous function. We apply a machine learning algorithm that involves a Gaussian process (GP) that measures the similarity between data points (i.e., the kernel function) to predict values for unseen points from training data. We use the generative model and treat fundamental parameters as given facts to predict observables. This gives us flexibility to prior fundamental inputs when modelling stars. We organise the rest of the paper as follow. Section \ref{sec:grid} contents descriptions about the computation of a representative stellar model grid. We then introduce the underline theory of GP and the setup of GP model in Section \ref{sec:gpmodel}. We then demonstrate some preliminary studies for low-demission problems in Section~\ref{examples}. Section \ref{sec:results} demonstrates GP predictions and their systematic uncertainties. Subsequently, we augment the grid to have a set of continuously-sampled stellar models and model 100 fake stars for testing the accuracy of our method in Section \ref{sec:augmentation}. Lastly, we discuss advantages and limitations of this approach, highlight areas where improvements can be found in the near future, and summary conclusions in Section \ref{sec:conclusion}.

% Set the context of the work.
% Cite relevant earlier studies

%% Lots of work on estimating stellar properties where observables are compared with stellar models.  Typical approach is grid based.  Lots of citations.  

%% Observables can come from all over.  Spectroscopic surveys (APOGEE, Galah, LAMOST, Gaia ESO, +), Astrometric Gaia, Photometric variability CoRoT, Kepler, K2, TESS, soon PLATO.

%% Lots of different models available with lots of different flavours - ask Tanda ...

%% Typical parameters to vary can refer to the star (mass, age, [Fe/H], Y_i) or they can refer to the model (MLT, overshoot, diffusion).  Most studies, certainly for field stars, treat all parameters as being independent.  

% Describe the problem we aim to solve

%% Plenty of work exists on HBM models in astro (cite fest).  By pooling together parameters we can win - for example EB's/cluster age, chemical comp.  But also we could pool parameters of the models MLT, Ov.  If we take a Bayesian approach the pooled constraint on MLT or Ov has the ability to constrain stellar parameters (e.g., age, mass).  The posterior distribution is a joint distribution!

%% Curent limitation is that this is all very tricky computationally.  Great news though - breakthroughs in machine learning, sampling methods, and GPU implementation means we now have a shot at doing this.  In this paper we give a deminstration of principle for one way of proceeding.

% Layout of this paper ...

%% 


\section{Theoretical stellar grid}\label{sec:grid}

We compute a stellar model grid as the training dataset. We aim to cover stars with approximate solar mass on the main-sequence and the subgiant phases. The mass range is set up as 0.8 -- 1.2$\rm M_{\odot}$. The computation of evolutionary tracks starts at the Hayashi line and terminates at the base of red-giant branch (RGB) where $\log g$ = 3.6 dex. Note that we only use models after the zero-age-main-sequence (ZAMS). We define ZAMS as the point where core-hydrogen burning contributes over 99.9\% of the total luminosity. 
%
The stellar gird considers four independent fundamental inputs which are stellar mass ($M$), initial helium fraction ($Y_{\rm init}$), initial metallicity ([Fe/H]$_{\rm init}$), and the mixing-length parameter ($\alpha_{\rm MLT}$). 
%
We calculated three model grids. First, a primary grid covers the whole input range. Uniform grid step is applied for $M$, $Y_{\rm init}$, $\alpha_{\rm MLT}$, and we use tow different grid steps for [Fe/H]$_{\rm init}$ below and above 0.2 dex.  Second, an additional grid is computed for $M$ > 1.05$\rm M_{\odot}$. Grid points of this grid are in between of the primary grid to increase the resolution for tracks with the 'hook'. 
Third, we compute off-grid models with random fundamental input values as an independent dataset for validating and testing GP models. 
%4,880 tracks with input parameters that are randomly sampled in the grid ranges for validating GPR models. The evolution time step was mainly controlled by the set-up tolerances on changes in surface effective temperature and luminosity. We also saved structural models for computing theoretical oscillation models.
%
Details of the computation are listed in Table \ref{tab:grid}. 

\begin{table}
	\centering
	\caption{Computation of Stellar model grid.}
	\label{tab:grid}
	\begin{tabular}{llll} % four columns, alignment for each
		\hline
		\multicolumn{3}{c}{Primary Grid}\\
		\hline
		Input Parameter & Range & Increment \\
        \hline
	$M$ ($\rm M_{\odot}$) & 0.80 -- 1.20 &  0.01\\
        $\rm{[Fe/H]}$ (dex) & -0.5 -- 0.2/0.2 -- 0.5 & 0.1/0.05\\
        	$Y_{\rm init}$ & 0.24 -- 0.32 & 0.02\\
        $\alpha_{\rm{MLT}}$  & 1.7 -- 2.5&  0.2\\
        \hline
       \multicolumn{3}{c}{Additional Grid}\\
	\hline
	Input Parameter & Range & Increment \\
        \hline
	$M$ ($\rm M_{\odot}$)  & 1.055 -- 1.195 &  0.01\\
        $\rm{[Fe/H]}$ (dex) & 0.25 -- 0.45 & 0.1\\
        	$Y_{\rm init}$ & 0.25 -- 0.31 & 0.02\\
        $\alpha_{\rm{MLT}}$  & 1.8-- 2.4&  0.2\\
        \hline
        \multicolumn{3}{c}{Off-grid Models}\\
        \hline
        \multicolumn{3}{c}{Input Parameters} &N\\
        \hline
         \multicolumn{3}{l}{Random $M$, [Fe/H]$_{\rm init}$ = 0.0, $Y_{\rm init}$ = 0.28, $\alpha_{\rm MLT}$ = 2.1 } & 44\\
        \multicolumn{3}{l}{Random $M$ and [Fe/H]$_{\rm init}$, $Y_{\rm init}$ = 0.28, $\alpha_{\rm MLT}$ = 2.1 }&174\\
        \multicolumn{3}{l}{Random $M$, [Fe/H]$_{\rm init}$, $Y_{\rm init}$, and $\alpha_{\rm MLT}$}&4880\\
	\hline
	\end{tabular}
\end{table}

%\subsection{Stellar models and input physics}\label{subsec:stellar_model}

We use the stellar code Modules for Experiments in Stellar Astrophysics
(\textsc{MESA}, version 12115) to construct stellar grids. 
\textsc{MESA} is an open-source stellar evolution package which is undergoing active development. 
Descriptions of input physics and numerical methods
can be found in \citet{2011ApJS..192....3P,2013ApJS..208....4P, 2015ApJS..220...15P}.
We adopted the solar chemical mixture [$(Z/X)_{\odot}$ = 0.0181]
provided by \citet{2009ARA&A..47..481A}. 
The initial helium fraction ($Y_{\rm init}$) and initial metallicity ($\rm{[Fe/H]_{init}}$) are independent inputs. 
The initial chemical composition is calculated with 
\begin{equation}
\log (Z_{\rm{init}}/X_{\rm{init}}) = \log (Z/X)_{\odot} + \rm{[Fe/H]_{init}}.  \\
\end{equation}
We use the \textsc{MESA} $\rho-T$ tables based on the 2005
update of OPAL EOS tables \citep{2002ApJ...576.1064R} and OPAL opacity
supplemented by low-temperature opacity \citep{2005ApJ...623..585F}. 
The grey Eddington $T-\tau$ relation is used to determine boundary conditions for modelling the atmosphere.
The mixing-length theory is implemented and the convection is adjusted by the mixing-length parameter ($\alpha_{\rm MLT}$).
We also apply the \textsc{MESA} predictive mixing scheme \citep{2018ApJS..234...34P,2019ApJS..243...10P}, which improves model structures at the convective boundary.  
Atomic diffusion of helium and heavy elements was also taken into account. MESA calculates particle diffusion and gravitational settling by solving Burger's equations using the method and diffusion coefficients of \citet{Thoul94}. 
%We considered 8 classes of species (H1, He3, He4, C12, N14, O16, Ne20, and Mg24) We considered 8 classes of species (H1, He3, He4, C12, N14, O16, Ne20, and Mg24)
We consider eight elements (${}^1{\rm H}, {}^3{\rm He}, {}^4{\rm He}, {}^{12}{\rm C}, {}^{14}{\rm N}, {}^{16}{\rm O}, {}^{20}{\rm Ne}$, and ${}^{24}{\rm Mg}$)
for diffusion calculations, and have the charge calculated by the MESA ionization module, which estimates the typical ionic charge as a function of $T$, $\rho$, and free electrons per nucleon from \citet{Paquette1986}.
The \textsc{MESA} inlist used for the computation is available on \url{https://github.com/litanda/mesa_inlist}.  

%\subsection{Oscillation models and seismic $\Delta \nu$}\label{subsec:seismo_model}

%Theoretical stellar oscillations were calculated with the \textsc{GYRE} code (version 5.1), which was developed by \citet{2013MNRAS.435.3406T}. And we computed radial modes (for $\ell$ = 0) by solving the adiabatic stellar pulsation equations with the structural models generated by \textsc{MESA}. We computed a seismic large separation($\Delta \nu$) for each model with theoretical radial modes to avoid the systematic offset of the scaling relation. We derived $\Delta \nu$ with the approach given by \citet{2011ApJ...743..161W}, which is a weighted least-squares fit to the radial frequencies as a function of $n$.  




\section{Gaussian Process Regression Model}\label{GPR}

GP models can be applied as probabilistic models to regression problems.  Here we will use the GP model to generalise a grid of stellar models to a continuous and probabilistic function that maps inputs (i.e., initial mass, chemical composition, etc.) to observable quantities (i.e., effective temperature, surface gravity, radius, etc.).  We aim to use the GP model as a non-parametric emulator, that is emulating the comparatively slow calls to models of stellar evolution. 
%
We adopted the a tool package named \textsc{GPyTorch}, which is a GP framework developed by \citep{gardner2018gpytorch}. GPyTorch is a Gaussian process library implemented using PyTorch. We adopted this tool package because it provides significant GPU acceleration, state-of-the-art implementations of the latest algorithmic advances for scalability and flexibility, and easy integration with deep learning frameworks. Source codes and detailed introductions could be found on \url{https://gpytorch.ai} 


\subsection{Gaussian Process Application}

We start with a grid of stellar models containing $N$ models with a label we want to learn, for example model effective temperature, which we will denote with the general symbol $\bf y$, and a set of input labels $\bf X$ (e.g., mass, age, and metallicity).  We can use a GP to make predictions of the effective temperature (or $y$) for additional input values given by $\bf X_{\star}$.  The vector $\bf y$ is arranged ${\bf y} = \left(y_{i}, ... ,y_{N} \right)^{T}$ where the subscript label references the stellar model.  The input labels are arranged into a $N \times D$ matrix where $D$ is the number of input dimensions (e.g., $D=3$ for mass, age, and metallicity) so that ${\bf X} = ({\bf x}_{1}, ..., {\bf x}_{N})^{T}$ where ${\bf x_{i}} = (x_{1, i}, ..., x_{D, i})^{T}$.  The matrix of additional inputs $\bf X_{\star}$ has the same form as $\bf X$ but size $N_{\star} \times D$.

Rasmussen and Williams (!REF!), from which our description below is based, define a GP as a collection of random variables, where any finite number of which have a joint Gaussian distribution.  In general terms, GP's may be written as
\begin{equation}
y({\bf x}) \sim \mathcal{GP}\left( m({\bf x}), {\bf \Sigma}\right),
\end{equation}
where $m({\bf x})$ is some mean function, and ${\bf \Sigma}$ is some covariance matrix.  The mean function controls the deterministic part of the regression and the covariance controls the stochastic part.  The mean function defined here could be any deterministic function and we will label the additional parameters, or hyperparameters, $\phi$.  Each element of the covariance matrix is defined by the covariance function or {\it kernel function} $k$ which has hyperparameters $\theta$ and is given by,
\begin{equation}
{\bf \Sigma}_{n, m} = k({\bf x}_{n}, {\bf x}_{m}),
\end{equation}
where the inputs ${\bf x}_{i}$ are $D$-dimensional vectors and the output is a scalar covariance.

As a GP is a collection of random variables, where any finite number of which have a joint Gaussian distribution, the joint probability of our data $\bf y$ is
\begin{equation}
p({\bf y} | {\bf X, \phi, \theta}) = \mathcal{N}(m({\bf X}), {\bf \Sigma}).
\end{equation}
If we want to obtain predictive distributions for the output $\bf y_{\star}$ given the inputs $\bf X_{\star}$ the joint probability distribution of $\bf y$ and $\bf y_{\star}$ is Gaussian and given by
\begin{equation}
p \left( \begin{bmatrix} {\bf y} \\ {\bf y_{\star}} \end{bmatrix} \right) = \mathcal{N} \left( \begin{bmatrix} {\bf X} \\ {\bf X_{\star}} \end{bmatrix} , \begin{bmatrix} {\bf \Sigma} & {\bf K_{\star} }\\ {\bf K_{\star}}^{T} & {\bf K_{\star \star}} \end{bmatrix}  \right), 
\end{equation}
where the covariance matrices $\bf \Sigma$ and $\bf K$ are computed using the kernel function so that
\begin{equation}
{\bf \Sigma}_{n, m} = k({\bf X}_{n}, \, {\bf X}_{m}),
\end{equation}
which is an $N \times N$ matrix.
\begin{equation}
{\bf K}_{\star \, n, m} = k({\bf X}_{n}, \, { \bf X}_{\star \, m}),
\end{equation}
which is an $N \times N_{\star}$ matrix, and finally
\begin{equation}
{\bf K}_{\star \star \, n, m} = k({\bf X}_{\star \, n},  {\bf X}_{\star \,m}),
\end{equation}
which is an $N_{\star} \times N_{\star}$ matrix.
The predictions of $\bf y_{\star}$ are again a Gaussian distribution so that,
\begin{equation}
{\bf y}_{\star} \sim \mathcal{N}(\bf \hat{y}_{\star}, \, \bf C),
\label{eq:pred}
\end{equation}
where 
\begin{equation}
{\bf \hat{y}}_{\star} = m({\bf X}_{\star}) + {\bf K}_{\star}^{T} \, {\bf \Sigma}^{-1} \, ({\bf y} - m(\bf X)),
\end{equation}
and 
\begin{equation}
{\bf C} = {\bf K}_{\star \star} - {\bf K}_{\star}^{T} \, {\bf \Sigma}^{-1} \, {\bf K_{\star}}.
\end{equation}

At point we can make predictions on model properties given a grid of stellar models using equation \ref{eq:pred}.  But these predictions will be poor unless we select sensible values for the form and hyperparameters of the mean function and covariance function.  In the following section we detail a number of kernel functions that will be tested against the data.  We will then discuss the method for determining the values of the hyperparameters to be used.

\subsection{GP Model Inputs and Outputs}
We aim to derive stellar parameters based on fundamental inputs of the model grid. As mentioned in Section~\ref{sec:grid}, our model grid has five independent fundamental inputs, says, mass, initial metallicity, initial helium fraction, mixing-length parameter, and the age.
Among them, the age ranges of individual evolutionary tracks significantly vary with the mass and other model inputs and hence are not ideal as the GP model inputs. The fractional age is an option. However, the evolution of global parameters around the 'hook' and the turn-off point performs sharp changes within a short time scale as shown in the left panel of Figure~\ref{fig:eep}. The sharp features are hard to learn and hence poorly predicted by GP models. To overcome this issue, we followed MIST(add ref here) and introduce an Equivalent Evolutionary Phase ($EEP$ here after) to replace the age as fundamental inputs. Note that our definition of $EEP$ is different from MIST. 
%
On each evolutionary track, we compute the displacement of a evolving step $n$ on the  $T_{\rm eff} - \log g$ as 
\begin{equation}\label{eq:disp}
\delta d_{n} = ((T_{\rm eff, n-1} - T_{\rm eff, n}) ^{2} + (\log g _{n} - \log g_{n-1})^{2}))^{f},
\end{equation}
and the total displacement of a data point $n$ 
\begin{equation}
d_{n} = \sum_{i = 0}^{i = n} \delta d_{i} .
\end{equation}
The normalised $d_{n}$ (to a 0 --1 range) is defined as $EEP$. On the same evolutionary track, $EEP$ equals to 0 at the ZAMS and 1 at $\log$ = 3.6 $dex$. The factor $f$ in Eq. \ref{eq:disp} is for modulating $EEP$ to avoid large gap in the parameter space, and we found that $f$ = 0.18 gives the best data distribution. {\bf add figures in appendix}
%
We illustrated two GPR models with the fractional age and $EEP$ as inputs in Figure~\ref{fig:selection_of_t} which shows the advantages of EEP.
It can be seen in the top graph that the effective temperature evolution presents a sharp hump around 0.7 (the hook) and quickly drops in the last 10\% lifetime. Using $EEP$ instead of fractional age significantly smoothes the curve and make it easer to fit. As we compare in Figure~\ref{fig:eep}, using EEP instead of fractional age gives a much smoother change at the hook and turn-off points. 

We summary our selections of GPR model inputs and outputs as below. 
\begin{itemize}
\item GPR model inputs and their dynamic ranges:
\item[] Mass ($M$ = 0.8 -- 1.2$M_{\odot}$)
\item[] Equivalent Evolutionary Phase ($EEP$ = 0 -- 1)
\item[] Initial metallicity ([Fe/H]$_{\rm init}$ =  -0.5 -- 0.5)
\item[] Initial helium fraction ($Y_{\rm init}$ = 0.24 -- 0.32)
\item[] Mixing-length parameter ($\alpha_{\rm MLT}$ = 1.7 -- 2.5)
\item GPR model outputs: 
\item[] Effective temperature ($T_{\rm eff}$) 
\item[] Surface gravity ($\log g$)
\item[] Radius ($R$)
\item[] The large spacing ($\Delta\nu$)  
\item[] Surface metallicity ([Fe/H]$_{\rm surf}$)
\item[] Age ($\tau$)
\end{itemize}
Thus, our GPR model can be described as 
\begin{equation}\label{gprmodel}
{\rm Outputs} = f(M, t', ({\rm Fe/H})_{\rm init}, Y_{\rm init}, \alpha_{\rm MLT}). 
\end{equation}

\begin{figure*}
        \includegraphics[width=1.\columnwidth]{2d_fage_data.pdf}
	\includegraphics[width=1.\columnwidth]{2d_EEP_data.pdf}
     \caption{Surface plots of effective temperature as a function of fractional age (left) and EEP(right). It can be seen that using EEP instead of fractional age as the input gives much smoother features at the hook and turn-off points.}
    \label{fig:eep}
\end{figure*}

\subsection{Training Procedure}\label{workflow}
  
\subsubsection{Data Selection}

We have three types of data for training, validating, and testing GP models. The model grid is apparently the training data and the off-grid models are divided into validating and testing datasets by half to half. Note that validating and testing data are not on same evolutionary tracks so they do not share any information. Validating dataset is for validating the GP model in the training process and is mainly used for early stopping, which is a form of regularisation for avoiding overfitting. We choose off-grid models but not on-grid models as validating data because the grid is equally spaced. Without additional information between grid points, an optimiser could either use a smooth function or a periodic function to fit the data and find no obvious differences in the likelihood. We describe how we use validating data in Section \ref{sec:training}. Lastly, testing dataset is independent on the training process and used for evaluating the final GP models. 

Because the computational and memory complexity exponentially increase with the number of data involved in the Gaussian Process, only a subset of the model grid can be adopted. Moreover, the data density are not same at different evolutionary stages due to the \textsc{MESA} step-control strategy: stellar models are dense at the main-sequence and the red-giant phases but quite sparse on subgiant stage. The sampling method is hence critical to the performance of GP models. Firstly, we want the data to uniformly cover the parameter space for the best efficiency. Secondly, we need to highly weight models at phases where sharp changes present, e.g., models around the hook and turn-off points.   
We tested a few methods and found that using the displacement ($\delta d_{n}$) defined in Eq.~\ref{eq:disp} to weight sampling meets the above two requirements. 

\subsubsection{Training, Validating, and Testing GP Models}\label{sec:training}

We train GP models as a regression problem. We develop our training process based on two standard examples (Simple GP Regression and Stochastic Variational GP Regression) in the \textsc{Gpytorch} packages. We test different approaches to determine the training method. Details are described and discussed in the next Section. In the training process, we validate the GP model in every iteration and save it as the best model when the validation index decreases. The training terminate when the validating results stop improving for 300 iterations. Lastly, the best GP model is tested with the testing dataset.

Here we discuss the method for validating and testing a GP model. We do not use popular methods, such like RME because the GP model performances are not uniform across the whole parameter space due to different evolutionary features. We illustrate this in Figure \ref{fig:2dtest} with a 2D GP model for $T_{\rm eff}$. As it can be seen that, the function for the hook area ($M \geq 1.05 M_{\odot}$ and $EEP \leq 0.7$) is more complex than that for other smooth regions. The evolutionary features in this particular area are hence relatively difficult to learn by the GP model, leading to obvious different accuracy levels (as shown in the bottom graph).
%
For other output parameters, surface gravity and radius are similar to the case of effective temperature: obviously differences between the 'hook' and other area. The evolutionary feature of surface metallicity is not significantly affected by the hook, but the testing errors of its GP model are obviously higher at the early subgiant phase for high-mass tracks. This is because high-mass tracks maintain shallow convective envelope and hence have strong diffusion effect during the main-sequence stage.  Their surface metallicities at then end of the main-sequence are generally much lower than the initial value. When high-mass models evolve to the early subgiant phase, the quick expansion of the surface convective envelope brings back the settling heavy elements to the surface, resulting in a sharp raise of the surface metallicity.  When there is a sub-region in the parameter space where GP models always peforme worse than other areas, a RME value is not suitable. We hence need a validation method that reflects the general accuracy of a GP model as well as its performance in these particular regions.   
%
To find a proper method for validation, we examine the error distribution as shown at the bottom of Figure \ref{fig:2dtest}. For the case of effective temperature, errors of most loss-mass data follows a Gaussian distribution. Data in the hood-affected regions is $\sim10\%$ of the total sample. This is a small proportion hence does not significantly affect the main feature but form long tails on both sides.  
%
For validating the general accuracy, a standard deviation is suitable. The long tail could be quantity by two cumulative values: one at 95\% indicates the median error of the these values, the other at 99.7\% to describe the scatter. Thus, we use the sum of three cumulative values (at 68\%, 95\%, and 99.7\%) of absolute validation errors as an error index to qualify a GP model in validating and testing progresses. For the case in Figure \ref{fig:2dtest}, cumulative values at 68\%, 95\%, and 99.7\% are 1.07, 4.86, and 11.05$K$, which give an error index equals to 16.98. 

\begin{figure}
	% To include a figure from a file named example.*
	% Allowable file formats are eps or ps if compiling using latex
	% or pdf, png, jpg if compiling using pdflatex
	\includegraphics[width=1.0\columnwidth]{2d_GPmodel_function.pdf}
	\includegraphics[width=1.0\columnwidth]{2d_testing_hist_effective_T.pdf}	
    \caption{Top: The 2D GP model for $T_{\rm eff}$. Bottom: probability distributions of validating errors of the GP model. }  
    \label{fig:2dtest}
\end{figure}


\subsection{Primary Tests and Settings of Training}

Before training the grid, we need to set up the GP model including mean function, kernel, likelihood function, loss function, and the optimiser. As there are not many relevant studies, we test how different options affects the GP models and decide which to use. These primary tests are done with simple 2-demission (2D) GP models, which have two inputs: $M$ and $EEP$ and can be described as $p = f(M, EEP)$. 
%
Data for training 2D models (2D data here after) are selected  from the primary grid with fixed [Fe/H]$_{\rm init}$ (0.0), $Y_{\rm init}$ (0.28), and $\alpha_{\rm MLT}$ (2.1). The training dataset contents 41 evolutionary track and 24,257 data points.
%
We also computed 44 evolutionary tracks with same input [Fe/H]$_{\rm init}$, $Y_{\rm init}$, and $\alpha_{\rm MLT}$ but random $M$ for validating and testing the GP models. 

The 2D model can be trained with the Exact GP approach because the training data size is approximate 20,000. We start with the \textsc{Simple GP Regression} example (\url{https://docs.gpytorch.ai/en/stable/examples/01_Exact_GPs/Simple_GP_Regression.html})  to develop our training script.

\subsubsection{Mean Function}

We start with testing different mean functions. As demonstrated in Figure~\ref{fig:eep}, the map between model inputs and outputs show various of features in different regions of the parameter space: models with relatively high mass have complex curvatures around the hook and the turn-off point while other models follow smooth functions. Although the GP model does not significantly affected by the choice of the mean function due to the flexibility of kernels, we find that a simple mean function (constant or linear) leads to a long time for training and a significant increase of the complexity of kernels. We hence applied a more complex mean function which is flexible enough to map smooth as well as curved features. A Neural Network is well suitable. To cover higher-demissions GP model, we adopt an architecture includes 6 hidden layers and 128 nodes per layer. All layers apply the linear transformation to the incoming data. We apply the element-wise function (Elu) because it give relatively smooth mean functions. (add ref for NN and Elu)

\subsubsection{Likelihood and Loss Function}

We then test for the likelihood and loss function. Our training object is a theoretical model grid, there is hence no observed uncertainty for each data point. However, a tiny random uncertainty exists due to the approximations in the \textsc{MESA} numerical method. The noise model can be assumed as a Gaussian function with a very small deviation. We hence applied the standard Gaussian Likelihood function in \textsc{GPyTorch}. 
%   
A Likelihood  specifies the mapping from latent function values $f(X)$ to observed labels $y$.
We adopt the the standard likelihood for regression which assumes a standard homoskedastic noise model whose conditional distribution is
\begin{equation}\label{eq:likelihood}
p(y|f(x)) = f + \epsilon, \epsilon \sim \mathcal{N}(0, \sigma^{2}),
\end{equation}
where $\sigma$ is a noise parameter. 
%
Given the random uncertainty is small, we set up a small and fixed noise parameter and run a few tests. However, we found that it makes the GP models hard to converge and sometimes lead to obvious overfitting (poor behaviour between grid). When we set up this noise parameter free and start with a large initial value, it reduces to a small number anyway in the training progress because it is data-driven. For these reasons, we decide not put strict constraint for or prioritise this noise parameter and let the data determine. In practice, we set up a loose upper limit ($\sigma$  < 0.1) for the noise parameter. However, it should be note that a GP model with a larger noise parameter can not be a proper description for the training data even if it gives good validating or testing errors. Because of this, we only use GP model with a noise parameter $\lesssim 10^{-4}$.   
The the loss function is simply the exact marginal log likelihood.

\subsubsection{Optimiser}

With a Neural Network mean function and Gaussion likelihood function, we then run tests to decide the optimiser. We mainly compare two optimisers named SGD and Adam. Here SGD refers to Stochastic Gradient Descent, and Adam is a combination of the advantages of two other extensions of stochastic gradient descent, specifically, Adaptive Gradient Algorithm and Root Mean Square Propagation. 
%
The SGD optimiser in the \textsc{Troch} packages involved the formula from \citet{sutskever2013importance}, which makes it possible to train using stochastic gradient descent with momentum uses a well-designed random initialisation and a particular type of slowly increasing schedule for the momentum parameter. The application of momentum in SGD could improve its efficiency and make it less likely to stuck in local minimums. On the other hand, the Adam optimiser includes the \textsc{AMSGrad} variant developed by \citet{47409} to improve its weakness in the convergence to an optimal solution. With these new developments, the two optimisers give very similar results. We finally choose Adam because it works more efficiently and stable than the SGD.  
%
We adaptive learning rate instead of a fixed value. All training starts with a learning rate of 0.01 and decreases by a factor of 2 when the loss value stop reducing in the previous 100 iterations.    

\subsubsection{Early stopping}
We set up an early stoping method to determine when to terminate the training progress based on the validation index introduced in Section~\ref{sec:training}. In our training progress, we validate the GP model every iteration and save the current model if the validating results is by far the best. When an optimal solution is found, the validating errors stop decreasing and start increasing at some point when overfiiting occurs. To be efficient and prevent overfitting, we terminate the training progress when the validating index does not decrease in the previous 300 iterations. The last saved GP model will be adopted.   

\subsubsection{GP kernels}

With the above settings, we lastly test for selecting the best kernel for this work. We involved four basic kernels the tests as listed below:
\begin{itemize}
\item RBF: Radial Basis Function kernel (also known as squared exponential kernel)
\item RQ: Rational Quadratic Kernel (equivalent to adding together many RBF kernels with different lengthscales)
\item Mat12: Matern 1/2 kernel (equivalent to the Exponential Kernel)
\item Mat32: Matern 3/2 kernel 
\end{itemize}
We applied each basic kernel and numbers of combinations (RBF + Mat21, RQ + Mat21, Mat32 + Mat21, RBF + Mat32, RQ + Mat32) to train the 2D GP models. Among these kernels, the combined one RBF+Mat21 give the best fit to the training data, however, it does not give the best predictions for the validating and testing data (off-grid).  On the other hand, the GP model with the Mat32 kernel gives good (but not the best) fit for the training data, but the best accuracies for validating and the testing data. 
%
Comparing between the two kernels, the RBF+Mat12 Kernel is a combination of a smooth and a spiky kernel, which offers enough flexibility fit most features in the training data. However, the spiky function well fitting the sharp features will be applied in the nearby off-grid regions and this causes poor predictions. 
%
The smoothness of Mat32 is somewhere between a spiky (Mat21) and a smooth kernel (RBF). For the sharp changes, it gives a relatively smooth function compared with the Mat21 kernel and hence perform a good balance between on- and off-grid data. 
%
In the final models, we hence adopted Mat32 kernels for all trainings.  

 \subsection{Strategy for Large sample}

The model grid we aim to train contents about 10,000,000 data points, which is much more than the upper limit of data size of the Exact GP (20,000). We hence consider other GP approaches. Our tests for the large-sample strategy are carried out with a 3-demission (3D) GP model, which have three fundamental inputs: $M$ and $EEP$ and  [Fe/H]$_{\rm init}$. The 3D model can be described as  ($p = f(M, EEP, {\rm [Fe/H]_{init}})$). 
Data for training 3D models (3D data hereafter) are from the primary grid with fixed $Y_{\rm init}$ (0.28), and $\alpha_{\rm MLT}$ (2.1). The training sample include 562 tracks and $\sim$ 300,000 data points. 
%
For validating and testing purposes, we computed 174 evolutionary tracks with the same input $Y_{\rm init}$, and $\alpha_{\rm MLT}$ but random $M$ and [Fe/H]$_{\rm init}$ within grid ranges. 


We first consider the Stochastic Variational GP (SVGP) with \textsc{GPyTorch ApproximateGP} module and test with the example on \url{https://docs.gpytorch.ai/en/v1.1.1/examples/04_Variational_and_Approximate_GPs/SVGP_Regression_CUDA.html}. SVGP is an approximate scheme rely on the use of a series of inducing points which can be selected in the parameter space. It trains using minibatches on the training dataset and build up kernels on the inducing numbers. Underline principles and detailed descriptions of this approach can be found in \citet{hensman2015scalable}. The advantage of SVGP is the large capacity of sample size, however, the kernel complexities is still limited by the amount of inducing numbers. If SVGP uses inducing numbers as many as the training data of the Excat GP, better prediction accuracy will be expected because more information is given. Unfortunately, the the GPU memory is limited. When more training data is loaded, the inducing number for SVGP can not go up to 20,000 in our training. On the same GPU, a SVGP model is able to use up to10,000 inducing number when 100,000 training data is loaded. This is to say, although a SVGP model involves more data but sacrifices the kernel complicity compared with an Exact GP model. 
%
Our comparison between a SVGP model (with 10,000 inducing number with 100,000 training data) 
and an Exact GP model (with 20,000 training data) shows that the SVGP framework does not improve the GP model performance. 
%
For instance, the testing errors of $T_{\rm eff}$ at 68\%, 95\%, and 99.7\% are 2.0, 5.8, and 15.7 $K$ (error index = 23.5$K$)  for Exact GP model and 2.2, 6.8, and 15.1 $(error index = 24.1)$ for the SVGP model. 


We then investigate another approach designed for the large dataset named Structured Kernel Interpolation (SKI GP). SKI GP was introduced by \citet{wilson2015kernel}. It produces kernel approximations for fast computations through kernel interpolation and is a great way to scale a GP up to very large datasets (100,000+ data points). We follow the example on \url{https://docs.gpytorch.ai/en/stable/examples/02_Scalable_Exact_GPs/KISSGP_Regression.html} to develop our script. We run a few tests to train a 3D SKI GP model with 100, 000 training data. Compare with the Exact GP and SVGP, its testing errors of $T_{\rm eff}$ are slightly improved to 2.0, 6.1, and 14.8 $K$ (error index = 22.9 $K$). However, the further test on the 5-demission data is not ideal: a SKI GP model with 100,000 training data performs much worse than an Exact GP model with 20,000 training data. The poor behaviour of the 5D model consist with what has been discussed in \citet{wilson2015kernel}. The method won’t scale to data with high dimensions, since the cost of creating the grid grows exponentially in the amount of data. We attempt to make some additional approximations with the \textsc{GpyTorch AdditiveStructureKernel} module. It makes the base kernel to act as one-dimension kernels on each data dimension. and the final kernel matrix will be a sum of these 1D kernel matrices. Although the testing errors are improved but still worse than the simple Exact model. Thus, the SKI GP framework is also not ideal for our goal.   

 \begin{figure}
	\includegraphics[width=1.0\columnwidth]{3d-testing_teff-10sections.pdf}
	\includegraphics[width=1.0\columnwidth]{3d-testing_teff-hist-10sections.pdf}	
    \caption{Top: Testing errors of 3D GP model for $T_{\rm eff}$ on the $M -- EEP$ diagram. Dashes indicates section boundaries. Bottom: examination of the edge effects of the section scenario. Probability distributions of testing errors of all testing data and those near the boundary ($\pm$0.01 EEP) in the upper graph are compared.  As it can be seen, testing errors do not raise around the boundary. }  
    \label{fig:3dtest}
\end{figure}


As mentioned above, the GPU memory captivity limits the actual number of data that induce the kernel and hence the accuracy level.  
This limit become critical for high-demission cases, because the parameter space exponentially increases with model demission and hence GP model accuracy inevitable declines. Thus, a simple way of improving GP models is using more data to induce kernels. 
%
For this reason, we break the grid into a number of sections and train GP models for each section separately. With the same 20,000 training data for a single GP model, 10 sections means 10 times of the data are used to describe the grid. 
%
The downside of this section scenario is obvious: there will not be one GP model that maps the whole grid. However, as long as the goal of this work is augmenting a model grid but not deriving a universal function for stellar evolutions, this scenario is suitable. 

Here we test how this approach works for the 3D data. We divide the dataset into 10 equal segments by $EEP$. We train one Exact GP model with 20,000 training data for each output parameter in each section. We then use the same testing dataset as used above to quantify predictions for the five outputs, and the error index are averagely improved by around 10\%. For instance, the error index of $T_{\rm eff}$ decreases from 23.5 to 21.6 (1.7/5.0/14.9 $K$ at 68/95/99.7\%). 
%

The section scenario method gives a better accuracy as expected, but there is a major concern about the edge effects at the boundary between sections. If the model works significantly poor at the boundary between sections, it will be difficult to map the systematic errors across the whole parameter space. We examine this in Figure~\ref{fig:3dtest}. We show absolute testing errors for $T_{\rm eff}$ on the $M -- EEP$ diagram (the top panel) and no obvious edge effects can be seen. We also do a statistical comparison between all data and those around the boundary ($\pm0.01EEP$) as demonstrated at the bottom. The density distributions of the two datasets are very similar. We hence conclude that there is no obvious edge effect.
%
The section scenario is adopted in the following study.  Our set up for the GP model is summarised as below:
\begin{itemize}
\item Model Type: Exact GP with the section scenario
\item Kernel: Mat32 (for all outputs)
\item Mean Function: Neural Network with 6 linear layers x 128 nodes and element-wise function (Elu) 
\item Likelihood Function: Gaussian Likelihood Function
\item Loss Function: Exact marginal $\log$ likelihood
\item Optimiser: Adam including AMSGRAD variant
\item Early Stoping: determined by the validating error index
\end{itemize}







     







\section{Training Results}\label{sec:results}

We train GP models using the method described in Section \ref{sec:gpmodel}. The training data are evolutioanry tracks in both primary and additional girds as mentioned in Table~\ref{tab:grid}. The role of the additional grid is increasing the grid resolution for evolutionary tracks with the 'hook'. 
%As discussed in the previous section, the kernel function in the 'hook' area are much more complex than other regions , and hence a GP model requires more information to map the feature in this region. 
The total training data includes $\sim$15,000 evolutionary tracks ($\sim$10,000,000 stellar models). Off-grid tracks are split 50-to-50 for validating (in the training progress) and testing (after the training progress) GP models.
The section scenario is applied. For each section, we train a GP model for each output parameter. Each training involves 20,000 training data points and 20,000 validating data points. For the testing dataset, we sample 50, 000 data points in the whole $EEP$ range. Note that we do not use models with $\tau \geq$ 20.0 Gyr, [Fe/H]$_{\rm surf} \leq$ -0.6 dex, or $T_{\rm eff} \geq$ 7000$K$ in testing dataset. 

\subsection{Overview of Results}

We start with training one \textsc{Exact GP} model for the whole $EEP$ range as a standard. Testing errors for this model are listed in Table~\ref{tab:results}. Compared with 2D and 3D cases, testing errors remarkably rise with increasing demission. For example, EI for $T_{\rm eff}$ goes up to 46K, which is much higher than that for the 2D (16K) or the 3D model (24K). 

We then train GP models with the section scenario. We gradually increase the number of sections and track down the changes in testing errors. 
We find that testing EI is not significantly improved when the grid are divided by more than 10 sections. We list the testing errors for different cases in Table~\ref{tab:results}. As it shown that, testing EIs for 10-, 20-, and 100-sections cases are very close. It turns out that the 10-sections case is the most efficient strategy and we take this case as the best result. Following analysis are all based on it. 

A overview of testing errors can also be seen in Figure~\ref{fig:5d_test_vs_input}, where we demonstrate rolling medians and rolling standard deviations of testing errors as a function of input parameters.
%
Median values are approximate along zero in most plots, indicating good agreement between GP predictions and true values.
%
The 68\% confidential intervals are generally small and do not significantly vary. However, the 95\% confidential intervals vary in large dynamic ranges and clearly depend on $M$, $EEP$, and $\rm [Fe/H]_{init}$. The long tails in marginal distributions are similar to what is illustrated in Figure~\ref{fig:2dtest}: GP predictions are relatively poor in some particular regions. Because the systematic uncertainty are not uniform through the parameter space, marginal error distributions do not well describe the systematic uncertainties. We hence investigate systematic uncertainty in a local scale. 

\begin{table*}
	\centering
	\caption{Training and validating errors for GPR Models}
	\label{tab:results}
	\begin{tabular}{cccccccccc}
		\hline
		Model Type&Inputs&$N_{\rm Training}$ &Sampling rate &\multicolumn{5}{c}{Testing Errors (at 68/95/99.7\%)} \\
		 \hline
		 \multicolumn{4}{c}{}& $T_{\rm eff}$ &$\log g$  &$R$  &[Fe/H]$_{\rm surf}$   &$\tau$ \\
		 \multicolumn{4}{c}{}&  (K)& ($10^{-3}$dex) & ($10^{-3}R_{\odot}$)  &  ($10^{-3}$dex)  & ($10^{-2}$Gyr) \\		 
		 \hline
		 Exact GP & 2D & 20,000 x 1 &96\% & 1/5/11 & 1/3/8 & 2/6/14 & 0.5/2/12 &  1/3/9 \\
		 %SVGP & 2D & 2,000 x 10  &96\% & 1/5/11 & 1/4/10 & 3/7/14  & 0.3/2/14 & 1/4/10  \\
		 \hline		 
		 Exact GP & 3D & 20,000 x 1 & 5\% & 2/6/16 & 1/4/10 & 3/7/17 &  2/6/22 & 2/7/22 \\
		 %Exact GP (5 sections) & 3D & 20,000 x 5 & 20\% & 2/6/17 &1/4/10 & 3/7/16& 1/3/18& 2/6/19\\
		  Exact GP (10 sections) & 3D & 20,000 x 10 & 50\% & 2/5/15 &1/4/11 & 2/7/17& 1/3/20& 2/6/19\\
		% SVGP (too slow) & 3D & 2,000 x 50 & 25\% & 3/7/16 & \\
		 %SVGP (too slow)  & 3D & 5,000 x 20 & 25\% & 2/7/15 & & & & & \\
		 %GIGP (Relu + normal data)& 3D & 100K & 25\% & 2/7/14  & & & & & \\
		 %GIGP (Relu + normal data)& 3D & 200K & 50\% & 4/8/17  & & & & & \\
		 %GIGP (Relu + grid data)& 3D & 220K & 50\% & 5/10/28 & & & & & \\
		 %GIGP (elu + grid data)& 3D & 220K & 50\% & 7/12/29 & & & & & \\
		 %\hline
		 %Exact GP (multitask) & 3D & 20,000 x 1 & 5\% & memory  &  &  &   &  \\
		  \hline		 
		 Exact GP & 5D & 20,000 x 1 & 0.2\% & 3/9/34 & 2/5/18 & 4/11/36 & 2/7/30 & 3/9/27  \\
		 Exact GP (3 sections) & 5D& 20,000 x 3 & 0.6\% &  3/8/27 & 2/5/18 & 3/7/26 & 1/4/24 &3/7/22 \\
		 %Exact GP (3 sections) & 5D& 20,000 x 3 & 0.6\% &  2.5/7.7/27.4 & 15/45/177 & 26/73/257 & 9/42/236 &26/73/226 \\
		 Exact GP (5 sections) & 5D& 20,000 x 5 & 1\% &  2/7/25 & 1/4/15 & 3/7/24 & 1/4/21 &2/6/22 \\
		 %Exact GP (5 sections) & 5D& 20,000 x 5 & 1\% &  2.4/7.2/24.8 & 13/39/152 & 25/71/242 & 9/39/207 &21/64/215 \\
		 Exact GP (10 sections) & 5D& 20,000 x 10 & 2\% & 2/7/27  & 1/4/14 & 2/7/26 &1/4/20 & 2/6/21\\
		 %Exact GP (10 sections) & 5D& 20,000 x 10 & 2\% & 2.4/7.1/26.8  & 14/40/141 & 24/70/263 &9/37/196 & 21/64/208\\
		 Exact GP (20 sections) & 5D& 20,000 x 20 & 4\% & 2/7/26  & 1/4/14 & 2/7/27 &1/3/18 & 2/6/22 \\
		 %Exact GP (20 sections) & 5D& 20,000 x 20 & 4\% & 2.2/6.9/26.1  & 13/38/141 & 24/72/290 &9/32/182 & 19/61/218 \\
		  Exact GP (100 sections) & 5D& 20,000 x 100 & 20\% & 2/7/25  & 1/4/14 & 2/7/26 &1/3/17 & 2/6/18 \\
		 % Exact GP (100 sections) & 5D& 20,000 x 100 & 20\% & 2.2/6.7/25.8  & 13/36/157 & 23/67/283 &8/34/185 & 19/59/190\\
		 %GIGP (Elu) & 5D & 10 x10,000 & 1\% & 5/12/34 & & & & \\
		 %GIGP (Relu) & 5D & 10 x10,000 & 1\% & 6/13/47 & & & & \\
		 %GIGP & 5D &  & 2\%, 5\%, 10\%, 25\%  & memory & & & & \\
		% \hline
		 % \multicolumn{9}{c}{GP a subset of 5D data-hook}\\
		 %\hline
		 %Exact GP EEP = 0.2-0.4 & 5D & 20,000 x 1 & 1\% & 2/7/20 &&&\\
		 %Exact GP EEP = 0.3-0.4 & 5D & 20,000 x 1 & 2\% & 3/8/21  &&&\\
		 %Exact GP EEP = 0.3-0.35 & 5D & 20,000 x 1 & 3\% & 2/7/17 &&&\\
		  %Exact GP EEP = 0.3-0.32 & 5D & 20,000 x 1 & 8\% & 2/5/14 &&&\\
		 %Exact GP EEP = 0.3-0.31 & 5D & 20,000 x 1 & 16\% & 2/4/13  &&&\\
		  \hline
	\end{tabular}
\end{table*}

\begin{figure*}
	\includegraphics[width=2.0\columnwidth]{ 5d-testing_vs_inputs.pdf}
    \caption{ Roll medians and 68/95\% confidential intervals of testing errors against GP model  inputs. Black solid lines indicate the median value; grey and blue shadowes represent the 68\% and 95\% confidential interval. Testing errors of $T_{\rm eff}$, $\log g$, and $R$ mainly depend on $M$ and $EEP$. Metallicity error strongly depends on $M$, $EEP$, and [Fe/H]$_{\rm init}$, and age error has a significant correlation to $EEP$ and [Fe/H]$_{\rm init}$. However, testing errors do not obviously relate to $Y_{\rm init}$ or $\alpha_{\rm MLT}$. } 
  \label{fig:5d_test_vs_input}
\end{figure*}
%

\subsection{Mapping Systematical Uncertainties using another GP model}\label{sec:sys}

As shown in Figure~\ref{fig:5d_test_vs_input}, systematical uncertainties relate to $M$, $EEP$, and [Fe/H]$_{\rm init}$ but not to $Y_{\rm init}$ or $\alpha_{\rm MLT}$.  Thus, a comprehensive model for the systematical uncertainty can be described as a function of $M$, $EEP$, and [Fe/H]$_{\rm init}$. In the $M$-$EEP$-$\rm [Fe/H]_{init}$ space, we divide this 3D space into equal-size segments and examine local error distributions. 
The choice of segment size matters. It needs to be small enough for only presenting the local feature, but it can not be very small so that there are not enough data points for proper statistical analysis. 
%
To find an appropriate size, we apply a statistic test. The purpose of the test is to check wether the local distribution has a tail feature. The condition we apply is either the ratio between 68\% and 95\% confidential intervals less than 2.5, or the ratio between 68\% and 99.7\% confidential intervals less than 4. 
%
After several attempts, we decide to divide the input range into 40 equally spaced segments for $M$, 50 for $EEP$, and 20 for [Fe/H]$_{\rm init}$. Hence there are 43,911 (41 x 21 x 51) gird points. We compute a rolling standard deviation for each grid point by using the data in a 3-segments (1.5 previous and 1.5 after) range across each demission. The statistic test shows that $\sim 90\%$ points meeting the above condition. 

We present local systematic uncertainties for $T_{\rm eff}$ ($\sigma T_{\rm eff}$) on the $M-EEP$ diagram in Figure \ref{fig:5d_sys_teff}. As it can be seen that, local $\sigma T_{\rm eff}$ values are mostly below $\sim$4K in the low-mass regions and raise up when mass is greater than 1.05$\rm M_{\odot}$. More important, there are substructures randomly appear in the parameter space. 
We visually inspect local systematic uncertainties for all output parameters and find similar features. 
%The substructures are expected because this is the residual of a flexible kernel function. 
Because of the existence of substructures, it is not convinced to describe the systematic uncertainty with any simple mathematical expressions. We hence use another GP model (GP-SYS model here after) to map three fundamental inputs ($M$, $EEP$, and [Fe/H]$_{\rm init}$) to local systematic uncertainty of each output parameter.   

Here we discuss how we set up the GP-SYS model. 
There are 43,911 data points in total. We randomly select 32,000 data points as the training dataset and use the rest 11,311 data points as the validating dataset. Because the training data size exceeds the 20,000 limit for \textsc{Exact GP}, we use other approach for the large data sample.
%
As discussed in Section \ref{sec:gpmodel}, the SVGP framework can be applied for when the kernel function is not very complex. It hence suitable for training the systematic uncertainty. We use 10,000 inducing points which are randomly selected in the 3D space. The training data is split into 10 batches for the SVGP training. 
%
We use a constant mean function and the RBF kernel because the data distribution is smooth. The variational evidence lower bound (ELBO) is adopted as the loss function. This approach is designed for when there is too much data for the exact inference. We set up Early Stopping by tracking the RMSE value of validating data. The training progress terminates when the RMSE value stops decreasing for 30 iterations.
The likelihood function and the optimiser are same as those for training GP-Grid models. 
Our set up for the GP-SYS model is listed as follow.
\begin{itemize}
\item Model Type: SVGP with 10,000 inducing number. 
\item Model Inputs: $M$, $EEP$, and [Fe/H]$_{\rm init}$
\item Model Outputs: $\sigma_{T_{\rm eff}}$, $\sigma_{\log g}$, $\sigma_{R}$, $\sigma_{\rm [Fe/H]_{surf}}$, and $\sigma_{\tau}$.
\item Training dataset: 3200 x 10 data points
\item Validating dataset: 11,311 data points
\item Kernel: RBF (for all outputs)
\item Mean Function: Constant Mean Function 
\item Likelihood Function: Gaussian Likelihood Function
\item Loss Function: The variational evidence lower bound (ELBO) 
\item Optimiser: Adam including AMSGRAD variant
\item Early Stoping: when validating RMSE stops decreasing for 30 iterations
\end{itemize}    


\begin{figure}
	\includegraphics[width=1.0\columnwidth]{5d_sys_teff.pdf}
	\includegraphics[width=1.0\columnwidth]{5d_sys_effective_T_std_predictions.pdf}
    \caption{Local systematic uncertainty (1-$\sigma$) for $T_{\rm eff}$ on the $M - EEP$ diagram for [Fe/H]$_{\rm init}$ = 0.0. The actual distribution is on the top and GP predictions are presented at the bottom.} 
  \label{fig:5d_sys_teff}
\end{figure}

We train GP-SYS models with the above method. Final models show good agreement with validating data. For instance, the average validating error for $\sigma_{T_{\rm eff}}$ is only 0.15K. For other output parameters, we summary the results in Table \ref{tab:sys}.
%
Figure \ref{fig:5d_sys_teff} includes a comparison between the actual and the GP-trained $\sigma_{T_{\rm eff}}$. It shows that the GP-SYS model well reproduces the $\sigma_{T_{\rm eff}}$ distributions. 

\begin{table}
	\centering
	\caption{Residual of GP models for systematic uncertainty}
	\label{tab:sys}
	\begin{tabular}{lc}
		\hline
		GP output& Average Validating Errors (ABS) \\
		\hline
		$\sigma_{T_{\rm eff}}$  (K) & 0.15 \\
		$\sigma_{\log g}$  ($10^{-3}$dex)   & 0.08 \\
		$\sigma_{R}$ ($10^{-3}R_{\odot}$)   & 0.2 \\
		$\sigma_{\rm [Fe/H]_{\rm surf}}$ ($10^{-3}$dex) & 0.09 \\
		$\sigma_{\tau}$ ($10^{-2}$ Gyr)  & 0.2\\
		%$\sigma_{\Delta\nu} (\mu Hz)$ & 0.01\\
		  \hline
	\end{tabular}
\end{table}


\section{Augmenting the MESA Grid}\label{sec:augmentation}

\subsection{A GP-Trained Model Set}

Now we have GP-Grid models for predicting observable outputs and GP-SYS models for estimate their local systematical uncertainties. 
In Figure~\ref{fig:5d_augmentation}, we demonstrate a GP-trained Kiel diagram comparing with the stellar grid. It shows that GP has transformed the sparse model grid into a non-sparse model set. 
%
Here we generate a GP-Trained model set to augment the stellar grid. We randomly sample 5,000,000 model data points with a flat distribution on each input demission. We then predict observable outputs with GP-Grid models and systematic uncertainties with GP-SYS models. 
%
This GP-trained model set hence includes five fundamental inputs ($M$, $EEP$, [Fe/H]$_{\rm init}$, $Y_{\rm init}$, and $\alpha_{\rm MLT}$), five outputs, ($T_{\rm eff}$, $\log g$,  $R$,  [Fe/H]$_{\rm surf}$, and  $\tau$), and five systematic uncertainties ($\sigma_{T_{\rm eff}}$, $\sigma_{\log g}$,  $\sigma_{R}$,  $\sigma_{\rm [Fe/H]_{\rm surf}}$, and $\sigma_{\tau}$). This model set can be downloaded at \url{a-place-for-data}. 

\begin{figure*}
	\includegraphics[width=1.3\columnwidth]{5d-au-mass.pdf}
	\includegraphics[width=0.7\columnwidth]{5d-au-mass-sys.pdf}
	%\includegraphics[width=1.2\columnwidth]{5d-au-feh.pdf}
	%\includegraphics[width=0.65\columnwidth]{5d-au-feh-sys.pdf}
	%\includegraphics[width=1.2\columnwidth]{5d-au-y.pdf}
	%\includegraphics[width=0.65\columnwidth]{5d-au-y-sys.pdf}
	%\includegraphics[width=1.2\columnwidth]{5d-au-alpha.pdf}
	%\includegraphics[width=0.65\columnwidth]{5d-au-alpha-sys.pdf}
    \caption{ Left: a GP-trained Kiel diagram compared with the stellar grid. Right: GP-predicted systematical uncertainties for all GP-trained models.} 
  \label{fig:5d_augmentation}
\end{figure*}

\subsection{Modelling fake stars with GP-trained models}

With the GP-trained model set, we model 100 fake stars to examine the accuracy of our method. Fake stars are randomly selected from the off-grid models. We use four observables, i.e., $T_{\rm eff}$, $\log g$, $R$, and [Fe/H]$_{\rm surf}$, as constraints. We apply typical observed uncertainty that is $\pm$50K for $T_{\rm eff}$ (high-resolution spectroscopy), $\pm0.005$dex for $\log g$ (seismology), $3\%$ for $R$ (seismology), and $\pm0.05$dex for [Fe/H]$_{\rm surf}$ (high-resolution spectroscopy). 
%
To avoid edge effect, fakes stars are selected in the range of $T_{\rm eff}$ = [4700K, 6800K], $\log g$ = [3.7, 4.6], [Fe/H]$_{\rm surf}$ = [-0.35,0.35], $M$ = [0.85,1.15], $EEP$ = [0.05,0.95], $Y_{\rm init}$ = [0.25,0.31], and $\alpha_{\rm MLT}$ = [1.8,2.4].  

We fit fake stars using the Maximum Likelihood Estimate (MLE) method. Note that the error term in MLE formula contents observed and systematic uncertainties given by GP-SYS models ($\sigma^{2} = \sigma_{obs}^{2} +  \sigma_{sys}^{2} $).
%
We present likelihood distributions of inferred stellar parameters for a representative fake star in Figure \ref{fig:fit_comparison}. The results based on grid modelling are also plotted as comparisons. 
%
Observed constraints for this fake star are $T_{\rm eff}$ = 4926$\pm$50K, $\log g$ = 4.536$\pm$0.005, [Fe/H]$_{\rm surf}$ =  0.34$\pm$0.05, and $R$ =  0.829$\pm$0.025$R_{\odot}$. True values of fundamental stellar parameters are $M$ = 0.861 $\rm M_{\odot}$, $\tau$ = 10.8 Gyr, $\rm [Fe/H]_{init}$ = 0.403, $Y_{\rm init}$ = 0.281, and $\alpha_{\rm MLT}$ = 2.356. 
%
Compared with the stellar grid, GP-trained model set has a completed statistical sampling and hence gives more sensible posteriors. The improvement for the age is obvious. It is under-sampled in the model grid and hence the inferred age does not actually converge. The Gird-based modelling infers an age of $7.7^{+3.2}_{-4.2}$Gyr, while GP-based modelling gives $8.3^{+2.6}_{-2.8}$Gyr. Compared with the true value (10.8 Gyr), GP determines a more accurate and precise result than the grid. For initial metallicity, initial helium fraction, and the mixing-length parameter, GP makes it possible to statistically estimate these parameters and inferred $\rm[Fe/H]_{init}$ and $Y_{\rm init}$ well agree with true values. The mixing-length parameter is not constrained because no correlated observable is given. This comparison clearly shows the advantage of GP-based modelling. It overcomes the under-sampling issue of a sparse grid and improve the accuracy and precision of estimates. 

\begin{figure*}
	\includegraphics[width=1.9\columnwidth]{gp_fitting.pdf}
    \caption{Probability distributions of estimated fundamental parameters from grid-based (Top) and GP-based modelling (Bottom) for a fake star. Observed constraints for this fake star are $T_{\rm eff}$ = 4926$\pm$50K, $\log g$ = 4.536$\pm$0.005, [Fe/H]$_{\rm surf}$ =  0.34$\pm$0.05, and $R$ =  0.829$\pm$0.025$R_{\odot}$. True values of fundamental parameters are $M$ = 0.861 $\rm N_{\odot}$, $\tau$ = 10.8 Gyr, $\rm [Fe/H]_{init}$ = 0.403, $Y_{\rm init}$ = 0.281, $\alpha_{\rm MLT}$ = 2.356, which are represented by blue dashes.} 
  \label{fig:fit_comparison}
\end{figure*}

We now examine the accuracy of inferred mass and age of GP-based modelling. This can be done by comparing the offset (truth - estimated value) with the estimated uncertainty. If the offset is average zero and its scatter consists with typical estimated uncertainty, it would indicate that the offset is just the random error.   
%
We make marginal likelihood distributions for each fake star and measure the 16th, 50th, and 84th percentile values to estimate the mass and the age. The comparison between true and estimated values is demonstrated in Figure \ref{fig:fake_test}. Mass offsets are around zero and the scatter range (0.032$\rm M_{\odot}$) is smaller than the typical estimated uncertainty (0.04$\rm M_{\odot}$). Age differences also has a mean value at approximate zero. The scatter is relatively large compared with the case for mass but still reasonably consistent with the estimated uncertainty: offsets of 96 fake stars are within 1-$\sigma$ (1.6 Gyr) and the rest 4 stars slightly exceeds up to 1.7Gy. The results infer that this GP-based modelling gives reasonable accurate estimates when modelling real stars.  

\begin{figure*}
	\includegraphics[width=1.8\columnwidth]{fake-stars-test.pdf}
    \caption{Differences between true and estimated masses (Top) and ages (Bottom) of 100 fake stars. Count distributions of offsets are demonstrated on the right side.} 
  \label{fig:fake_test}
\end{figure*}











\section{Discussion and Conclusions}\label{sec:conclusion}

Discussions: advantages (GP give continuous functions, it is efficient (1 hour/model) , section scenario offers flexibility for large grid); limitations, .... future work: GP individual stars, MCMC etc. 

Conclusions: GP can be an accurate and efficient approach for augmenting stellar model grid.   

\section*{Acknowledgements}

Development of GPyTorch is supported by funding from the Bill and Melinda Gates Foundation, the National Science Foundation, and SAP.

%%%%%%%%%%%%%%%%%%%%%%%%%%%%%%%%%%%%%%%%%%%%%%%%%%

%%%%%%%%%%%%%%%%%%%% REFERENCES %%%%%%%%%%%%%%%%%%

% The best way to enter references is to use BibTeX:

\bibliographystyle{mnras}
\bibliography{ref} % if your bibtex file is called example.bib


%%%%%%%%%%%%%%%%%%%%%%%%%%%%%%%%%%%%%%%%%%%%%%%%%%

%%%%%%%%%%%%%%%%% APPENDICES %%%%%%%%%%%%%%%%%%%%%

%\appendix
%\onecolumn
%\section{Validation and prediction of GPR models }

\subsection{GPR model with 3-D inputs}
\begin{itemize}
\item Training set 
\item validation 
\item GP predictions
\end{itemize}



\subsection{GPR model with 5-D inputs}
\begin{itemize}
\item Training set 
\item validation 
\item GP predictions
\end{itemize}
\begin{figure}
	% To include a figure from a file named example.*
	% Allowable file formats are eps or ps if compiling using latex
	% or pdf, png, jpg if compiling using pdflatex
    \caption{Validations for 5D inputs GPR models before and after chunking.}
    \label{fig:4d_vali1}
\end{figure}





%%%%%%%%%%%%%%%%%%%%%%%%%%%%%%%%%%%%%%%%%%%%%%%%%%


% Don't change these lines
\bsp	% typesetting comment
\label{lastpage}
\end{document}

% End of mnras_template.tex